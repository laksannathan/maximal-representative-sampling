\section{Terminology and Definitions}

A well-defined learning problem involves a number of design choices, including selecting the type of training experience, the target function to be learned, a representation for this target function, and an algorithm to learn from the source of training experience [2]. In modelling a political participation process, a computer program is designed to approximate the likelihood of a person going to vote on election day. Ideally, for every instance with unknown political interest and willingness to participate, there is enough data of people of similiar demographics, socioeconomics and psychological traits to generalize from. This chapter defines key terminology and destinctions when learning from biased data. Basic design issues and approaches to supervised learning are covered, while conceptual elements of interest are introduced with regards to overfitting. The role of noise in the bias-variance decomposition will be analyzed and further broken down. sources of error.

\subsection{Sampling Bias}

Sampling bias is often referred to as selection bias or sample selection bias. I will stick to the more descriptive term sampling bias. It underlines the fact that the bias arises in how the data was sampled. Also, the use of the term becomes less ambiguous, because there exists another notion of selection bias in the context of model selection. This type of bias is usually referred to as bad generalization, where the performance of the selected hypothesis is overly optimistic. [input: convenience sampling]

Although one could employ a census to measure the entire population, it is more common to take a sample of the population. A properly designed probability sample (see probability sampling) can be used to make estimates for not only the sample itself, but also for the underlying population from which it was selected. A probability sample is one in which each element of the (underlying) population has a known and non-zero chance of being selected. That is, every person has a chance to be included in the study and have his or her characteristics, opinions, etc., become part of the data. It should be noted that everyone does not have to have an equal chance of being selected – just a known non-zero chance of being selected. 
Probability samples have several desirable characteristics. They enable us to put a margin of error or confidence interval on our estimates – essentially a measure of how accurate the estimate is compared to the same estimate calculated on the full population. Probability samples make it possible to not only compare the sample to the population, but also to compare a sample from one population to a sample from another population,

\subsection{Representative Sample}

Some examples include sex, age, education level, socioeconomic status or marital status. Information collections with biased tendencies can't generate a representative sample.

Variables considered in the study must accurately reflect the populations characteristics. 

Consider \textit{attribute: income} of a subset of GBS participants. Statistical significance tests, e.g. Kolmogorov-Smirnov, Chi-Squared

\begin{figure}[ht]
	\begin{center}
		\includegraphics[scale=0.40,angle=0]{fig/tree3}
		\label{project}
		\caption{.}
	\end{center}
\end{figure}

Given a subset of GBS, similarity scores can be defined to evaluate the distance to reference distributions from GESIS. Kolmogorov-Smirnov tests or Chi-Squared assess the likelihood of an attribute of GBS  There are \(2^{|GBS|} = 2^{587}\) subsets of GBS. Evaluating every possible combination of GBS participants and its score is computationally intractable.

A well-defined learning problem where large polls (Umfragedaten) may contain valuable implicit regularities, requires a well-specified task, performance metric and source of training experience [2]. The MRS problem is now stated as a binary classification task with GESIS as positive class and GBS as negative class. Consider designing a computer program to learn to distinguish between . Using prior knowledge together with past experience to guide learning, a machine learning algorithm
is fed with data from games that have been played by chess grandmasters. From this information, the program will learn to apply certain functions to specific board states and make decisions about which move to play next.

Consider a randomly chosen survey participant, i.e. an instance of GBS or GESIS. If the poll indicates the or

Descriptive statistics can be used to 
 
No practical amount of data can distinguish between two distributions, thus instances of GBS can not be proven to come from GESIS. However, discriminative learning allows to infer the conditional probability of \textit{'instance of GBS/GESIS'} given the survey data within a probabilistic framework:

Discriminative learners will look for decision boundaries to distinguish the different views of GBS from GESIS. False negatives are then more closely aligned with the target probability distribution. The process of classification is repeated until the learner starts fitting noise more than is warranted. To avoid overfitting, the learning objective needs to be refined as contingency tables lack proper interpretibility. Given the imbalanced nature and size of GBS, learning is restrained to simpler algorithms with lesser degrees of freedom. The fraction of false positives in the result set of this procedure is kept as proxy measure for the subsequent method positive-unlabeled learning (PU learning). The development of classification models in this setting is often referred to as positive-unlabeled learning (Denis et al. 2005).

\vspace{15pt}
\begin{figure}[ht]
	\begin{center}
		\includegraphics[scale=0.48,angle=0]{fig/roc_example}
		\label{project}
		\caption{There is no caption for such a stupid figure.}
	\end{center}
\end{figure}

PU learning is a semi-supervised technique that does not make the simplifying assumption of GBS instances being negative. Instead, a one-class classifier is trained on GESIS only. [...] This can result in even better assessment. [Read Literature] - Imporance weighted cross validation and pu learning with proper assessment.

\subsection{The Problem of Overfitting}

The model in supervised learning usually refers to the mathematical structure of how to make predictions \(y_i\) given \(x_i\). The most common model is a linear regression model, where the prediction is given by a linear combination of weighted input features. The parameters, the weights of these features, are the undetermined part that need to be learned from data. Depending on the task, the prediction value can have different interpretations, i.e. regression or classification. The categorical outcome, "did vote" or "did not vote", makes political participation a binary classification problem [7]. In machine learning, the terms hypothesis and model are often used interchangeably. This paper uses the following convention [3] as the terminology to describe ideas and concepts is not standardized:

\begin{itemize}
\item The phrase single hypothesis refers to a single probability distribution or function. An example is the polynomial \(2x^2 + 3x + 1\).

\item The word model refers to a set of probability distributions or family of functions with the same functional form. An example is the set of all quadratic functions.

\item As a generic term, hypothesis refers to both single hypotheses and models.
\end{itemize}

With the definitions above, it is a hypothesis selection problem if both the degree of a polynomial and the corresponding parameters are of interest. The phrase single hypothesis refers to a single probability distribution or function. A machine learning model, the composite hypothesis, refers to a family of probability distributions or functions with the same functional form. An example is the set of all second-degree polynomials [9].

%\subsection{Bias-Variance Tradeoff}

"Overfitting is the disease. Noise is the cause. Learning is led astray by fitting the noise more than the actual signal"[6]. To avoid overfitting you might deliberately exclude certain factors, increase sample size, stop the analysis early, or simply pick less complex algorithms. Regularization puts a break where additional iterations of algorithms start to harm the performance. Validation is another way to see what will actually happen out-of-sample

By definition, statistical inference is taking the results of applying some sort of construct or model to specific data and then speculating that it would continue to perform well beyond the original observation range. Given a set of training samples \((x_i,y_i)\) find a single hypothesis \(h\) that "fits the data well": \(y_i = h(x_i)\) for most \(i\). The equation is characterized by a trade-off between goodness-of-fit and complexity of the hypothesis:

\begin{itemize}
\item if \(h\) is too simple, \(y_i = h(x_i)\) may not hold for many values of \(i\);
\item if \(h\) is too complex, it fits the data very well but will not generalize well on unseen data.
\end{itemize}

%\subsection{Stochastic and Deterministic Noise}

