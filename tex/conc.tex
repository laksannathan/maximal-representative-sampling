\chapter{Conclusion}\label{Sec:Conclusion}


Imbalanced setting where the minority class GBS is the unlabeled class of interest but handled as negative. evaluation in the absence of actual negatives is further enhanced by class imbalance. puF-Measure instead of (or in addition to) puROC could reduce the effects of class imbalance. inclusion of f1-measure would be interesting. finally, the label "representative" is actually a property of a sample and not its single instances. survey mismatch from gbs to gesis renders most attributes useless regarding entropy. forces valu comparisons essentially introduce non existent pattern. one class classification suffers from high variance in estimating the fraction of actually representative gbs data. assessment of binary classifiers in pu settings does not lead to more accurate roc curve estimations as the fraction of positives cannot be estimated. overrepresentative underrep. rep and nonrep. trhoughout the thesis are not properly defined.

at the analysis stage or the manuscript writing stage, this leads to time-consuming and cost-inefficient rechecking of data, redoing analyses, and rewriting of the manuscript. The following is a list of issues that IDA may detect that show the possible importance of such detection:

This combination of class imbalance with non-stationary environments poses significant and interesting practical problems for classification


Numerical summaries of distributions
A distribution can be summarized with various descriptive statistics. The mean and median capture the center of a distribution (central tendency) while the variance describes the distribution spread or variability (see online book material).  
Mean: the average of a number of values. It is calculated by adding up the values and dividing by the number of the values (how many the values there are). 

Median: The "median is the number separating the higher half of a data sample, a population, or a probability distribution, from the lower half" (Reviews, 2013). For a highly skewed distribution, the median may be a more appropriate measure of central tendency than the mean. For example, the median is more widely used to characterize income, since potential outliers (e.g., those with very high incomes) have much more impact on the mean.
 
Variance: Variance is a measure of the extent to which a set of numbers are "spread out".

Precision: Precision is the reciprocal of the variance and is most commonly seen in Bayesian analysis (see Guideline 9).
% [http://ccsg.isr.umich.edu/index.php/chapters/statistical-analysis-chapter#nine]