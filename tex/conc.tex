\chapter{Conclusion}\label{Sec:Conclusion}

for relating GBS samples to GESIS. 
Although model learning and performance evaluation in a supervised setting are well understood, the availability of unlabeled data gives additional optionsand also presents new challenges.
Imbalanced setting where the minority class GBS is the unlabeled class of interest but handled as negative. evaluation in the absence of actual negatives is further enhanced by class imbalance. puF-Measure instead of (or in addition to) puROC could reduce the effects of class imbalance. inclusion of f1-measure would be interesting. finally, the label "representative" is actually a property of a sample and not its single instances. survey mismatch from gbs to gesis renders most attributes useless regarding entropy. forces valu comparisons essentially introduce non existent pattern. one class classification suffers from high variance in estimating the fraction of actually representative gbs data. assessment of binary classifiers in pu settings does not lead to more accurate roc curve estimations as the fraction of positives cannot be estimated. overrepresentative underrep. rep and nonrep. trhoughout the thesis are not properly defined.

%This paper presents a theoretical analysis of sample selection bias correction. The sample bias correction technique commonly used in machine learning consists of reweighting the cost of an error on each training point of a biased sample to more closely reflect the unbiased distribution. This relies on weights derived by various estimation techniques based on finite samples. We analyze the effect of an error in that estimation on the accuracy of the hypothesis returned by the learning algorithm for two estimation techniques: a cluster-based estimation technique and kernel mean matching. We also report the results of sample bias correction experiments with several data sets using these techniques. Our analysis is based on the novel concept of <em>distributional stability</em>which generalizes the existing concept of point-based stability. Much of our work and proof techniques can be used to analyze other importance weighting techniques and their effect on accuracy when using a distributionally stable algorithm. C. Cortes, M. Mohri, M. Riley, and A. Rostamizadeh. Sample selection bias correction theory. In Proceedings of the International Conference on Algorithmic Learning Theory, 2008.

Im Rahmen dieser Bachelorarbeit wurde ein Datensatz subsampled um für die
Die total nonresponse kann zwar nicht behoben werden, allerdings konnte der gesamte Datensatz repräsentativer gemacht werden, in dem Gruppen der Bevölkerung die in GBS überrepräsentiert waren aussortiert wurden. Durch wiederholtes Entfernen der GBS Teilnehmer, welche am besten aussortiert werden konnten, wurde bezüglich der verwendeten Attribute eine AUROC Verringerung von 0.2 erreicht. Ein Klassifizierer würde auf den verbliebenen GBS Instanzen eine AUROC von ungefähr 0.6 erreichen, wobei eine AUROC von 0.5 die Aussage implizieren würde, dass keine Unterscheidungen zwischen den Gruppen beider Klasse, das bedeutet GESIS und GBS, gemacht werden konnte. 
Die Verteilungen der Umfragen wurden von seiten GBS aneinander angepasst und das Covariate-shift Setting dadurch in Bezug auf zukünftige Analysen verbessert.

GBS Teilnehmer Discriminative Lerner konnten in keinem Bereich vermehrt GBS Teilnehmer erkennen. Diese Idee wurde in einen Positive-unlabeled Kontext gesetzt, womit eine bessere Einschätzung dieser AUROC möglich war. 

Normalerweise vergewissert man sich, dass die erhobenen Daten repräsentativ sind. Allerdings kann dies oft nicht garantiert werden. Gerade in den Sozialwissenschaften ist es allerdings möglich externe Datenquellen zu konsultieren, um den Bias in den Daten zu reduzieren. Oftmals geschieht dies in Form von gewichteten Datenpunkten. 

As part of this bachelor thesis, a data set subsampled to be used for the
Although the total nonresponse cannot be corrected, the entire data set could be made more representative by sorting out groups of the population that were over-represented in GBS. By repeatedly removing the GBS participants that could be best sorted out, an AUROC reduction of 0.2 was achieved for the attributes used. A classifier would achieve an AUROC of approximately 0.6 on the remaining GBS instances, whereas an AUROC of 0.5 would imply that no distinctions could be made between the groups of both classes, i.e. GESIS and GBS. 
The survey distributions have been adapted by GBS to each other and the covariate shift setting has been improved for future analysis.

GBS participants Discriminative learners could not identify more GBS participants in any area. This idea was put into a positive-unlabeled context, which allowed a better assessment of this AUROC. 

Normally, one makes sure that the data collected are representative. However, this can often not be guaranteed. Especially in the social sciences it is possible to consult external data sources in order to reduce the bias in the data. This is often done in the form of weighted data points. 
In fact, this procedure can be best described by Iterative AUROC reduction for maximal representative subsampling.
