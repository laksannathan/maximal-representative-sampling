\begin{thebibliography}{9}

\bibitem{tuscher}
Kalisch R, Mueller MB, Tuescher O (2015) \textit{Advancing empirical resilience research.} \textbf{Behav Brain Sci.} 38:e128.

\bibitem{rammstedt}
Rammstedt, B. and John, O.P. (2007). \textit{Measuring personality in one minute or less: A 10-item short version of  the Big Five Inventory in English and German}. \textbf{Journal of Research in Personality}, 41, 203–212. 

\bibitem{candela}
Candela J, Sugiyama M, Schwaighofer A, et al.: \textit{Dataset Shift In Machine Learning.} \textbf{MIT Press}, Cambridge, Massachusetts,2009.

\bibitem{west}
West BT, Sakshaug JW, Aurelien GAS (2016) \textit{How Big of a Problem is Analytic Error in Secondary Analyses of Survey Data?} \textbf{PLoS ONE} 11(6): e0158120. https://doi.org/10.1371/journal.pone.0158120

\bibitem{heeringa}
Heeringa, S.G., West, B. T., and Berglund, P.A. (2010). \textit{Applied survey data analysis}. Boca Raton, \textbf{FL: CRC Press}.

\bibitem{likert}
Likert, R. (1932). \textit{A technique for the measurement of attitudes}. \textbf{Archives of Psychology}, 22(140).

\bibitem{likert3}
Jacoby, J, and Matell, M.S. (1971). \textit{Three-point Likert scales are good enough}. \textbf{Journal of marketing research}, 8(4), 495-500.

\bibitem{likert4}
Sullivan, G. M., and Artino, A. R. (2013). \textit{Analyzing and interpreting data from Likert-type scales}. \textbf{Journal of Graduate Medical Education}, 5(4), 541–542. 

\bibitem{denis}
 Francois Denis, Remi Gilleron, and Fabien Letouzey. \textit{Learning from positive and unlabeled examples.} \textbf{Theoretical Computer Science}, 348(1):70–83, 2005. 

\bibitem{elkan}
Charles Elkan and Keith Noto. \textit{Learning classifiers from only positive and unlabeled data.} In Proceedings of the 14th ACM SIGKDD international conference on Knowledge discovery and data mining, pages 213–220. \textbf{ACM}, 2008.

\bibitem{claesen}
Marc Claesen, Frank De Smet, Johan AK Suykens, and Bart De Moor. \textit{A robust ensemble approach to learn from positive and unlabeled data using svm base mode}, \textbf{Neurocomputing}, 160:73–84, 2015.

\bibitem{brick}
JM Brick, G. Kalton. \textit{Handling missing data in survey research }, 1996, https://doi.org/10.1177/096228029600500302.

\bibitem{rubin}
Rubin, D. B. \textit{Multiple imputation for nonresponse in surveys.}, 1987, New York: John Wiley and Sons.

\bibitem{bagging}
Leo Breiman. \textit{Bagging predictors.} \textbf{Machine learning}, 24(2):123–140, 1996.

\bibitem{tom}
Tom Mitchell: \textit{Machine Learning}, \textbf{McGraw-Hill}, 1997.

\bibitem{trevor}
Trevor Hastie, Robert Tibshirani, Jerome H. Friedman: \textit{Elements of Statistical Learning: Data Mining, Inference, and Prediction}, \textbf{Springer}, 2009.

\bibitem{ian}
Ian H. Witten, Eibe Frank: \textit{Data Mining: Practical Machine Learning Tools and Techniques}, \textbf{Morgan Kaufmann}, 2011.

\bibitem{yaser}
Yaser S. Abu.Mostafa, Malik Magdon-Ismail, Hsuan-Tien Lin: \textit{Learning From Data: A Short Course}, \textbf{AMLbook.com}, 2012.

\bibitem{stone}
Stone M.: \textit{Cross-Validatory Choice and Assessment of Statistical Predictions}, \textbf{Journal of the Royal Statistical Society}, 1976. 

\bibitem{tianqi}
Tianqi Chen and Carlos Guestin: \textit{XGBoost: A Scalable Tree Boosting System.} In Proceedings of the 22nd \textbf{ACM SIGKDD} International Conference on Knowledge Discoverry and Data Mining, 2016, pp.785794.

\bibitem{walker}
Walker, SH; Duncan, DB (1967). \textit{Estimation of the probability of an event as a function of several independent variables}. \textbf{Biometrika}. 54 (1/2): 167–178. doi:10.2307/2333860. JSTOR 2333860.

\bibitem{cox}
Cox, DR (1958). \textit{The regression analysis of binary sequences (with discussion)}. \textbf{J Roy Stat Soc B}. 20 (2): 215–242. JSTOR 2983890. 

\bibitem{smote}
Chawla NV, Bowyer KW, Hall LO, Kegelmeyer WP. \textit{SMOTE: synthetic minority over-sampling technique.} \textbf{J Artif Intell Res}. 2002;16:341–378.


\end{thebibliography}