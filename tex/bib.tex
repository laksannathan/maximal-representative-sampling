\begin{thebibliography}{9}

\bibitem{tuscher}
Kalisch R, Mueller MB, Tuescher O (2015) \textit{Advancing empirical resilience research.} \textbf{Behav Brain Sci.} 38:e128.

\bibitem{rammstedt}
Rammstedt, B. and John, O.P. (2007). \textit{Measuring personality in one minute or less: A 10-item short version of  the Big Five Inventory in English and German}. \textbf{Journal of Research in Personality}, 41, 203–212. 

\bibitem{candela}
Candela J, Sugiyama M, Schwaighofer A, et al.: \textit{Dataset Shift In Machine Learning.} \textbf{MIT Press}, Cambridge, Massachusetts,2009.

\bibitem{west}
West BT, Sakshaug JW, Aurelien GAS (2016) \textit{How Big of a Problem is Analytic Error in Secondary Analyses of Survey Data?} \textbf{PLoS ONE} 11(6): e0158120. https://doi.org/10.1371/journal.pone.0158120

\bibitem{heeringa}
Heeringa, S.G., West, B. T., and Berglund, P.A. (2010). \textit{Applied survey data analysis}. Boca Raton, \textbf{FL: CRC Press}.

\bibitem{likert}
Likert, R. (1932). \textit{A technique for the measurement of attitudes}. \textbf{Archives of Psychology}, 22(140).

\bibitem{likert3}
Jacoby, J, and Matell, M.S. (1971). \textit{Three-point Likert scales are good enough}. \textbf{Journal of marketing research}, 8(4), 495-500.

\bibitem{likert4}
Sullivan, G. M., and Artino, A. R. (2013). \textit{Analyzing and interpreting data from Likert-type scales}. \textbf{Journal of Graduate Medical Education}, 5(4), 541–542. 

\bibitem{roc}
Hanley JA, McNeil BJ (1982) \textit{The meaning and use of the area under a receiver operating characteristic (ROC) curve}. \textbf{Radiology} 143:29-36. 

\bibitem{jean}
Jean-Claude Deville and Carl-Erik Sarndal, \textit{Calibration Estimators in Survey Sampling}, \textbf{Journal of the American Statistical Association}
Vol. 87, No. 418 (Jun., 1992), pp. 376-382

\bibitem{shimodaira}
Hidetoshi Shimodaira, \textit{Improving predictive inference under covariate shift by weighting the log-likelihood function}, Journal of Statistical Planning and Inference 90 (2000) 227–244.

\bibitem{denis}
 Francois Denis, Remi Gilleron, and Fabien Letouzey. \textit{Learning from positive and unlabeled examples.} \textbf{Theoretical Computer Science}, 348(1):70–83, 2005. 

\bibitem{claesen}
Marc Claesen, Frank De Smet, Johan AK Suykens, and Bart De Moor. \textit{A robust ensemble approach to learn from positive and unlabeled data using svm base mode}, \textbf{Neurocomputing}, 160:73–84, 2015.

\bibitem{claesen2}
Claesen, M.; Davis, J.; De Smet, F.; and De Moor, B. 2015. \textit{Assessing binary classifiers using only positive and unlabeled data.} arXiv preprint arXiv:1504.06837.

\bibitem{jain}
S. Jain, M. White, and P. Radivojac. \textit{Recovering true classifier performance in positive-unlabeled learning}. In Proc. 31st AAAI Conference on Artificial Intelligence, \textbf{AAAI '17}, 2017.

\bibitem{jain2}
S. Jain, M. White, and P. Radivojac. \textit{Estimating the class prior and posterior from noisy positives and unlabeled data.} In Proc. 30th Advances in Neural Information Processing Systems, \textbf{NIPS '16}, pp. 2693–2701, 2016.

\bibitem{bickel}
Bickel, S. et al. (2009). Discriminative Learning Under Covariate Shift. Journal of Machine Learning Research, 10, 2137–2155

\bibitem{tom}
Tom Mitchell, \textit{Machine Learning}, 3rd Edition, \textbf{McGraw-Hill}, 1997.

\bibitem{Sechidis}
K Sechidis, B Calvo, and G Brown. \textit{Statistical hypothesis testing in positive unlabelled data}. In Machine Learning and Knowledge Discovery in Databases, pages 66–81. \textbf{Springer}, 2014

\bibitem{elkan}
C. Elkan and K. Noto, \textit{Learning Classifiers from Only Positive and Unlabeled Data}, in ACM SIGKDD International Conference on Knowledge Discovery and Data Mining. \textbf{ACM}, 2008, pp. 213–220.

\bibitem{brick}
JM Brick, G. Kalton. \textit{Handling missing data in survey research }, 1996, https://doi.org/10.1177/096228029600500302.

\bibitem{shimodaira}
Shimodaira, H. (2000). \textit{Improving predictive inference under covariate shift by weighting the log-likelihood function}. \textbf{Journal of Statistical Planning and Inference}, 90, 227–244.

\bibitem{leo}
Leo Breiman. \textit{Bagging predictors.} \textbf{Machine learning}, 24(2):123–140, 1996.

\bibitem{trevor}
Trevor Hastie, Robert Tibshirani, Jerome H. Friedman: \textit{Elements of Statistical Learning: Data Mining, Inference, and Prediction}, \textbf{Springer}, 2009.

\bibitem{ian}
Ian H. Witten, Eibe Frank: \textit{Data Mining: Practical Machine Learning Tools and Techniques}, \textbf{Morgan Kaufmann}, 2011.

\bibitem{yaser}
Yaser S. Abu.Mostafa, Malik Magdon-Ismail, Hsuan-Tien Lin: \textit{Learning From Data: A Short Course}, \textbf{AMLbook.com}, 2012.

\bibitem{jesse}
Jesse Davis , Mark Goadrich: \textit{The relationship between Precision-Recall and ROC curves}, Proceedings of the 23rd international conference on Machine learning, p.233-240, 2006, Pittsburgh, Pennsylvania  [doi>10.1145/1143844.1143874] 

\bibitem{walker}
Walker, SH; Duncan, DB (1967). \textit{Estimation of the probability of an event as a function of several independent variables}. \textbf{Biometrika}. 54 (1/2): 167–178. doi:10.2307/2333860. JSTOR 2333860.

\bibitem{cox}
Cox, DR (1958). \textit{The regression analysis of binary sequences (with discussion)}. J Roy Stat Soc B. 20 (2): 215–242. \textbf{JSTOR} 2983890. 

\bibitem{breiman2}
Leo Breiman (2001). \textit{Random Forests}. \textbf{Machine Learning}. 45 (1): 5–32. doi:10.1023/A:1010933404324. 

\bibitem{smote}
Chawla NV, Bowyer KW, Hall LO, Kegelmeyer WP. \textit{SMOTE: synthetic minority over-sampling technique.} \textbf{J Artif Intell Res}. 2002;16:341–378.

\end{thebibliography}