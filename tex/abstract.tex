%\section*{Abstract}
\chapter*{Abstract}
\thispagestyle{empty}
To allow statistical inference in social sciences, survey participants must be selected at random from the target population. When samples are drawn from parts of the population that are close to hand, subgroups might be over-represented. This leads to statistical analyses under sampling bias, which in turn may produce similarly biased outcomes. The present thesis uses machine learning to reduce this selection bias in a psychological survey using auxiliary information from comparable studies that are known to be representative. Discriminative algorithms are trained to directly characterize the divergence between representative and non-representative samples.


%Machine Learning models are trained to directly charaterize the divergence between representative and non-representative samples using auxiliary information from comparable studies that are known to be representative.

